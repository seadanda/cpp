\section{Introduction}
\label{sec:introduction}
This report aims to outline the design and implementation of a program written in \CC{} to calculate the impedence of alternating current (AC) circuits consisting of any number of ideal resistors, capacitors and inductors in series or parallel. The program attempts to make best use of the advanced idiomatic features of \CC.

%C++ intro
\CC{} is an object oriented programming language that was created by Bjarne Stroustrup in 1983, initially with the aim to introduce object oriented programming into C~\cite{Stroustrup:2000:CPL:518791} but which has since diverged to become a programming language in its own right and one of the most popular programming languages in the world.

Object oriented programming is centred around classes and their instantiations called objects which contain data and functions which can be either public, allowing the data or function to be accessed from outside of the class, or private, allowing access only within a specified scope. There is a third classification, protected, which will be discussed in section~\ref{sec:code}.

\subsection{AC circuit theory}
\label{sec:theory}
In contrast with DC circuit theory, where the current through a circuit is opposed only by the resistance, $R$, of that circuit, in AC circuit theory the current through an AC circuit is opposed by a combination of both the resistance and reactance, X and their combination is called the impedance, Z. Whereas resistance is the opposition to the flow of current, the reactance is the opposition to the \textit{change} in flow of current. This is why the reactance of components is irrelevant with DC circuits, as the current through the circuit is not changing.

\subsubsection*{Impedance}

Mathematically when dealing with AC circuits, complex numbers are used for the impedance with the real part representing the resistance and the imaginary part representing the reactance such that

\begin{equation}
  Z=R+jX ,
\end{equation}

where, to agree with convention, $j = \sqrt{-1}$ is used in place of $i$ to avoid confusion with the current which is conventionally denoted $I$~\cite{ref:omalley}.
The program is designed to make all of its calculations with ideal components, which means that a resistor only has resistance and zero reactance and that inductors and capacitors only have reactance and zero resistance. Therefore assuming ideal components, the impedance of a capacitor, inductor and resistor can be found using

\begin{align}
   Z_{R}&=R,   &   Z_{L}&=j\omega{}L,   & \textnormal{and} & &  Z_{C}&=\frac{1}{j\omega{}C}
\end{align}

respectively, where $\omega$ is the angular frequency of the AC current in $rad\,s^{-1}$ and $L$ and $C$ are the capacitance and inductance respectively~\cite{ref:omalley}.

With $n$ components connected in series, the total impedance of the circuit can be found by taking the sum of the impedances of the constituent components with

\begin{equation}
  \label{eqn:ser}
  Z_{total} = Z_1 + Z_2 + \ldots + Z_n
\end{equation}

and with $n$ components connected in parallel, the inverse of the total impedance can be found by taking the sum of the inverse of the impedances of the constituent components with

\begin{equation}
  \label{eqn:par}
  \frac{1}{Z_{total}} = \frac{1}{Z_1} + \frac{1}{Z_2} + \ldots + \frac{1}{Z_n}.
\end{equation}

Both equation~\ref{eqn:par} and equation~\ref{eqn:ser} also work for nested circuits, as each subcircuit essentially behaves like a component~\cite{ref:omalley}.

\subsubsection*{Phase difference}
%fact check this
The phase difference describes the offset between the voltage and the current. The phase difference is found by taking the argument of the impedance. Capacitors have a phase difference of $-90\degree$ which means that the voltage lags the current by $90\degree$~\cite{ref:cap} and inductors have a phase difference of $90\degree$ which means that the voltage leads the current by $90\degree$~\cite{ref:ind}.
